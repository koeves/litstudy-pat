% \section{Literature Review} \label{review}
% This section, describes the bodies of literature the paper is interested in. It begins by examining the main interest, namely the paper by Thiemann and Yamada\cite{thiemann}. Most attention is spent on this paper, their algorithm is introduced and its correctness and properties are discussed. Furthermore, some examples of certain are used to illustrate the behaviour of the algorithm.

% Following, the \textit{complement algorithm} by Lazrek et al. \cite{lazrek} is detailed, an algorithm to decide sufficient completeness (and indirectly also pattern completeness, as shown later). Finally, a tree automata-based construction is presented that can be used to decide pattern completeness. At the end of the section, further literature of interest is briefly listed.

\section{Thiemann and Yamada's algorithm} \label{thiemann-yamada}
The algorithm presented in the paper of Thiemann and Yamada \cite{thiemann} works on multisets of pattern problems and applies rules on the innermost matching problems, pattern problems and sets of pattern problems. Two iterations are presented, one dealing with only linear inputs (where no variable appears multiple times in the left-hand sides), and a further iteration with additional rules to handle non-linearity. The rules of the algorithm are presented here in a slightly different notation.

Matching problems (denoted as $mp$) are reduced using the following rules:
\begin{align*}
\textbf{decompose} & & \{(f(t_1, ..., t_n), f(l_1, ..., l_n))\} &\rightarrow \{(t_1, l_1), ..., (t_n, l_n)\} \\
\textbf{match} & & \{(t, x)\} \in mp &\rightarrow \varnothing \text{ if } \forall (t', l) \in mp \text{. } x \notin Var(l) \\
\textbf{clash} & & \{(f(...), g(...))\} &\rightarrow \bot_{mp}\text{ if }f \neq g
\intertext{For pattern problems (sets of matching problems – denoted as $pp$), the following rules apply:}
\textbf{remove-mp} & & \{\bot_{mp}\} &\rightarrow \varnothing \\
\textbf{success} & & \{\varnothing\} &\rightarrow \top_{pp}
\intertext{Finally for sets of pattern problems (which is the input of the algorithm, denoted as $P$), the rules are as follows:}
\textbf{failure} & & \{\varnothing\} &\rightarrow \bot_P \\
\textbf{remove-pp} & & \{\top_{pp}\} &\rightarrow \varnothing \\
\textbf{instantiate} & & \{pp\} &\rightarrow \text{Inst$(pp,x)$ if }\{(x, f(...))\} \in pp
\end{align*}

The \textbf{match} rule removes a matching problem from the set in case variable $x$ does not occur in any other matching problems in the same set. This intuitively mean, we match variable $x$ to $t$.

The last rule, \textbf{instantiate}, is applicable in case variable $x$ is tried to be matched with some non-variable term $f(...)$. In this case, we construct a new pattern problem $pp\sigma_{x,c}$ for each constructor in $\mathcal{C}$, given as:
$$pp\sigma_{x,c} = \{\{ (t\sigma_{x,c},l) \mid (t,l) \in mp\} \mid mp \in pp\}$$
where
$$\sigma_{x,c} = [x \mapsto c(x_1, ..., x_n)]$$ for each $c \in \mathcal{C}$ of arity $n$, and fresh variables $x_1, ..., x_n$.

In order the deal with non-linearity, further rules are introduced so that the algorithm does not get stuck on non-linear input. These rules are as follows:

\begin{align*}
\textbf{clash'} & & \{(t,x), (t',x)\} \in mp &\rightarrow \bot_{mp}\text{ if $t$ and $t'$ clash} \\
\textbf{instantiate'} & & \{\{(t,x), (t',x)\}\} \in P &\rightarrow \text{Inst$(pp,x)$} \\
\textbf{failure'} & & \{pp\} \in P &\rightarrow \bot_P
\end{align*}

In case of \textbf{instantiate'}, we can apply the rule if $t$ and $t'$ differ in variable $x$ of finite sort (such that $\{t \mid t : \iota \in \mathcal{T}(\mathcal{C})\}$ is a finite set).

In case of \textbf{failure'}, we need to fail the algorithm if within each $mp \in pp$ there exists $\{(t,x),(t',x)\}$ such that $t$ and $t'$ differ in variable $x$ of infinite sort. This special case is needed, as it would be impossible to instantiate a variable of infinite sort, if we find that terms differ in such variable. In case not all matching problems differ in such a variable, we can mark those problems locally as $\bot_{mp}$ (via \textbf{clash'}).

We say a term $t$ and $t'$ clash if $t|_p=f(...) \neq g(...)=t'|_p$. Terms $t$ and $t'$ differ in variable $x$ if $t|_p \neq t'|_p$ and $x \in \{t|_p, t'|_p\}$.

The order, in which the reduction steps are applied in the list-based implementation are given as follows:
\begin{enumerate}
    \item Exhaustively apply \textbf{decompose}, \textbf{clash} and \textbf{clash'}
    \item Exhaustively apply \textbf{match} and try to apply \textbf{failure'}
    \item Invoke \textbf{instantiate} or \textbf{instantiate'} with preference for the former
\end{enumerate}

\subsection{Examples} \label{examples-alg}

The following examples illustrate certain executions of the algorithm. We assume the sort of natural numbers with one defined symbol $f : \mathbb{N} \rightarrow \mathbb{N}$, using the following constructors $\mathcal{C}_{\mathbb{N}} = \{0 : \mathbb{N}, s(x) : \mathbb{N} \rightarrow \mathbb{N}\}$.

\begin{example} Linear case \label{ex-thiemann-lin}

Given a left-hand side of $\{f(0), f(s(x))\}$, the linear algorithm would compute:
\begin{align*}
    P &= \{\{\{(f(a), f(0))\}, \{(f(a), f(s(x)))\}\}\}\\
\text{decompose} &\Rrightarrow^{*} \{\{\{(a, 0)\}, \{(a, s(x))\}\}\} \\
\text{instantiate} &\Rrightarrow^{*} \{\{\{(0, 0)\}\}, \{\{(s(z), s(x))\}\}\} \\
\text{match} &\Rrightarrow \{\{\bot_{mp}\}, \{\{(s(z), s(x))\}\}\} \\
\text{remove-mp} &\Rrightarrow^{*} \{\{\varnothing\}, \{\{(s(z), s(x))\}\}\} \\
\text{decompose} &\Rrightarrow \{\{\varnothing\}, \{\{(z, x)\}\}\} \\
\text{match} &\Rrightarrow \{\{\varnothing\}, \{\varnothing\}\} \\
\text{success} &\Rrightarrow^{*} \{\top_{pp}, \top_{pp}\} \\
\text{remove-pp} &\Rrightarrow^{*} \varnothing
\end{align*}
\end{example}

The algorithm concludes that the left hand sides are pattern complete.

\newpage
\begin{example} Linear case failure \label{ex-thiemann-lin-fail}

The following example demonstrates how the algorithm would find an incomplete pattern:
\begin{align*}
    P &= \{\{\{(f(a), f(s(x))\}\}\}\\
\text{decompose} &\Rrightarrow \{\{\{(a, s(x))\}\}\} \\
\text{instantiate} &\Rrightarrow \{\{\{(0, s(x))\}\}, \{\{(s(z), s(x))\}\}\} \\
\text{clash} &\Rrightarrow \{\{\bot_{mp}\}, \{(s(z), s(x))\}\} \\
\text{remove-mp} &\Rrightarrow \{\varnothing, \{(s(z), s(x))\}\} \\
\text{failure} &\Rrightarrow \bot_{P}
\end{align*}
\end{example}

Here we could have further reduced the pattern problem $\{\{(s(z), s(x))\}\}$, however, do the incomplete case $\{\{(0, s(x))\}\}$ resulting in $\varnothing$, the whole pattern problem reduces to $\bot_{P}$.

\begin{example} \label{example-general} General case

Given defined symbol $f : \mathbb{N} \times \mathbb{N} \rightarrow \mathbb{N}$ the following left-hand sides:\\$\{\{f(x, x)\}, \{f(x, y)\}\}$ the algorithm would compute:
\begin{align*}
    P &= \{\{\{(f(a, b), f(x, x))\}, \{(f(a, b), f(x, y))\}\}\\
\text{decompose} &\Rrightarrow^{*} \{\{\{(a, x), (b, x)\}, \{(a, x), (b, y)\}\}\} \\
\text{clash'} &\Rrightarrow \{\{\bot_{mp}, \{(a, x), (b, y)\}\}\} \\
\text{remove-mp} &\Rrightarrow \{\{\{(a, x), (b, y)\}\}\} \\
\text{match} &\Rrightarrow^{*} \{\{\varnothing\}\} \\
\text{success} &\Rrightarrow^{*} \{\top_{pp}\} \\
\text{remove-pp} &\Rrightarrow^{*} \varnothing
\end{align*}
\end{example}

\vspace{0.2em}
\begin{example} Linear algorithm with non-linear input

The following example illustrates that the additional rules to match non-linear inputs are necessary to decide pattern completeness, using the non-linear left-hand side from Example \ref{example-general}:
\begin{align*}
    P &= \{\{\{(f(a, b), f(x, x))\}\}\}\\
\text{decompose} &\Rrightarrow^{*} \{\{\{(a, x), (b, x)\}, \{(a, x), (b, y)\}\}\}
\end{align*}
At this stage, the only rule we might want to apply is \textbf{match}. However, due to the condition in that rule that the variable $x$ cannot appear in any right hand side within the matching problem, we fail to apply the rule. Therefore, the algorithm is stuck, indeed no other rule apply (not even \textbf{instantiate}, since $x$ is not a constructor term). The correct step in this case would be to apply \textbf{clash'}, available in the extension of the algorithm to the general case.

\end{example}

\begin{example} Quasi-reducible LHS is not pattern complete \label{quasi-ex}

Let us apply the algorithm to Example \ref{quasi-intro}. In that example the following term rewriting system $\mathcal{R}_{\mathbb{Z}}$ is presented, using $$\mathcal{C}_{\mathbb{Z}} = \{\text{true}: \mathbb{B}, \text{false}: \mathbb{B}, 0: \mathbb{Z}, s: \mathbb{Z} \rightarrow \mathbb{N}, p: \mathbb{Z} \rightarrow \mathbb{N}\}$$ with one defined symbol $\mathcal{D}_{\mathbb{Z}} = \{\text{even}: \mathbb{Z} \rightarrow \mathbb{B}\}$:
\begin{align*}
    \text{even}(0) &\rightarrow \text{true} & \text{even}(s(0)) &\rightarrow \text{false} & \text{even}(s(s(x))) &\rightarrow \text{true} \\
    \text{even}(p(0)) &\rightarrow \text{false} &
    \text{even}(p(p(x))) &\rightarrow \text{even}(x) \\ 
    s(p(x)) &\rightarrow x & p(s(x)) &\rightarrow x
\end{align*}
The algorithm would compute:
\begin{align*}
    P = &\{\{\{(even(z), even(0))\}, \{(even(z), even(s(0)))\}, \\
    &\{(even(z), even(s(s(x))))\}, \{(even(z), even(p(0)))\}, \\
    &\{(even(z), even(p(p(x))))\}, \{(even(z), even(p(p(0))))\}, \\
    &\{(even(z), s(p(x)))\}, \{(even(z), p(s(x)))\}\} \\
    \Rrightarrow^{*} &\{\{\{(s(z), 0)\},  \{(s(z), p(x))\}\}, \{\{(p(z), 0)\},\{(p(z), s(x))\}\}\}
\end{align*}

Previous steps are omitted for brevity, the only problematic pattern problems are the ones above, all other cases are resolved with $\top_{pp}$. These cases, however, would result in clashes, therefore the whole pattern problem reduces to an empty set, marking $P$ as $\bot_{P}$.
\end{example}

\subsection{Analysis}
This sections details the correctness of the algorithm. Firstly, that the reduction steps have normal form $\{\varnothing, \bot_{P}\}$, secondly, that the algorithm terminates, by defining a measure of difference between pattern problems and showing that each step decreases in this order.

The normal forms of the reduction steps is one of $\{\varnothing, \bot_{P}\}$. The algorithm iteratively removes matching problems with \textbf{match} or marks them incomplete with \textbf{clash}. Using the definition of completeness in Section \ref{def-pattern-complete}, a pattern problem $pp$ is complete if for each constructor ground substitution $\sigma$, there exists a matching problem $mp \in pp$ that is complete. Therefore, the pattern problem is marked $\top_{pp}$ by \textbf{success} when the pattern problem reduces to $\varnothing$ by removing all clashed and matching problems via \textbf{remove-mp}. When the pattern problem reduces to the empty set we can conclude that at least one \textbf{match} has taken place, therefore, the pattern is complete. On the other hand, when no match has taken place, the matching problem reduces to $\{\bot_{mp}\}$. The notation here is bit subtle, due to the nested empty sets. Namely, what leads to a pattern problem being marked successful is an empty pattern problem $\{\varnothing\}$. Furthermore, what leads to the whole execution marked as failure is a pattern problem containing only $\bot_{mp}$. The reader can refer to the examples in Section \ref{examples-alg} for further clarification.

Furthermore, the algorithm is shown to be decidable and terminating \cite{thiemann}. The termination proof for the left-linear case is given by defining a measure of difference $\abs{\ell - t}$ between terms \((\ell, t)\) of a matching problem as:
\begin{itemize}
    \item $\abs{\ell - x}$ the number of function symbols in $\ell$
    \item $\abs{f(\ell_1, ..., \ell_n) - f(t_1,...,t_n)} = \sum_{i=1}^{n} \abs{\ell_i - t_i}$
    \item $\abs{\ell - t} = 0$ otherwise
\end{itemize}
This measure is lifted to pattern problems by: 
$$\abs{pp}_{\text{diff}} = \sum_{mp \in pp, (t, \ell) \in mp} \abs{\ell - t}$$
Then, the relation $\succ$ is defined for sets of pattern problems by extending $>$ to multisets $>^{\text{mul}}$ by: 
$$P \succ P' \iff \{\abs{pp}_\text{diff} \mid pp \in P\} >^{\text{mul}} \{\abs{pp'}_\text{diff} \mid pp' \in P'\}$$.
The relation $\succ$ is strongly normalizing, each step of the algorithm weakly decreases, while the instantiate rule strictly decreases with respect to $\succ$ \cite{thiemann}. 

\section{Complement algorithm} \label{lazrek}
The \textit{complement algorithm} presented by Lazrek, Lescanne, and Thiel in \cite{lazrek}, represents a mechanism to conclude whether in a TRS $\mathcal{R}$, a defined symbol $f$ (called operator in the paper) is convertible to a set of constructors, denoted as \textit{relative} completeness. The algorithm can also be used to decide quasi-reducibility and indirectly pattern completeness, as demonstrated by Example \ref{quasi-ex-alg}. The main idea behind the algorithm is to compute the \textit{complement} of matched terms, then iteratively check whether the complement can be further reduced or matched. The complement of a term $t$ means the ground terms of $t$ and the complement of $t$ equal the set of constructor terms.

The algorithm works on pairs of sets $M_i$ and $N_i$, each iteration $M_{i+1}$ and $N_{i+1}$ are constructed from their previous counterpart. The algorithm starts by setting $M_0$ as the set representing the left-hand sides of $\mathcal{R}$, and the set $N_0$ as the set of ground instances of $f$ of the form $f(z_1, ..., z_n)$ where $f \in \mathcal{D}$. The algorithm then iteratively tries to unify elements of $N_0$ with the elements of $M_0$ with a substitution $\sigma$. In case such a substitution is found, the matched elements $m \in M$ and $n \in N$ are removed from $M_i$ and $N_i$ and new sets $M_{i+1}$ and $N_{i+1}$ are constructed by:
\begin{align*}
    M_{i+1} &= M_{i-1} \setminus \{m\} \cup \{m\rho \mid \rho \in C(\sigma), m\rho \neq m\sigma\} \\
    N_{i+1} &= N_{i-1} \setminus \{n\} \cup \{n\rho \mid \rho \in C(\sigma), n\rho \neq n\sigma\}
\end{align*}

The complement of a substitution $\sigma$ defined as the set $C(\sigma)$ of all substitutions $\rho$ different from $\sigma$, having the same domain and mapping elements to complementary term. The complement of a term $t$ is defined as the finite set such that the ground terms of $t$ and the ground terms of the set of complement terms equal the set of constructor terms.

The algorithm continues until $M_{last}$ or $N_{last}$ are empty (i.e. the last pair of $M$ and $N$ sets), or no further unification is possible. If both $M_{last}$ and $N_{last}$ are empty, $f$ is convertible to the constructors ($f$ covers all input terms and/or the complement can be further reduced by the rules). If $M_{last}$ is empty but $N_{last}$ is not empty, $f$ is not defined on the terms in $N_{last}$.

The set $N_{last}$ is interesting, as it contains the patterns not matched by $f$. It can, therefore, be used in a functional programming setting, warning the user of incomplete patterns (and hinting on actual constructor patterns currently unmatched).

From these, quasi-reducibility is determined when either each input term is matched by the LHS of $f$ (both $M_{last}$ and $N_{last}$ are empty), or the elements of $N_{last}$ can further be reduced.

Pattern completeness is given in case both $M_{last}$ and $N_{last}$ are empty.

\subsection{Examples}
\begin{example} Let us take an example execution of the algorithm on the same input as Example \ref{ex-thiemann-lin}.

\begin{align*}
    M_0 &= \{f(0), f(succ(x))\} \\
    N_0 &= \{f(z)\} \\
    \Rightarrow M_1 &= \{f(succ(x))\} \\
    N_1 &= \{f(succ(z))\} \\
    \Rightarrow M_2 &= \varnothing \\
    N_2 &= \varnothing
\end{align*}

The sets $M_0$ and $N_0$ are the starting steps, we would like to check whether $f(z)$ is convertible with the left-hand sides of the system. In the first iteration the case of $f(0)$ and $f(z)$ are unified by the substitution $\sigma = \{z \mapsto 0\}$. Then we pick a substitution $\rho$ from the set of complement substitutions $C(\sigma)$, take $\rho = \{z \mapsto succ(z)\}$. In the last step, we take $\sigma = \{z \mapsto x\}$. Since both $M_2$ and $N_2$ are empty, the definition of $f$ is said to be \textit{relatively complete}.
\end{example}

\begin{example} The counterpart of Example \ref{ex-thiemann-lin-fail}:

\begin{align*}
    M_0 &= \{f(succ(x))\} \\
    N_0 &= \{f(z)\} \\
    \Rightarrow M_1 &= \varnothing \\
    N_1 &= \{f(0)\} \\
\end{align*}

In the first step we try to unify $f(succ(x))$ with $f(z)$. This we can do via $\sigma = \{z \mapsto succ(x)\}$. Then we take $\rho$ from $C(\sigma)$ as $\rho = \{z \mapsto 0\}$, after which we try to unify $f(succ(x))$ and $f(0)$, which fails. In $N_1$ we find the term that is not covered by the definition of $f$ (it is not further reducable), therefore $f$ is not relatively complete.
\end{example}

\begin{example} \label{quasi-ex-alg} Finally, the quasi-reducible system $\mathcal{R}_\mathbb{Z}$ from Examples \ref{quasi-ex-complete}, \ref{quasi-ex}:
\begin{align*}
    M_0 &= \{even(0), even(s(0)), even(s(s(x))), even(p(0)), even(p(p(x)))\} \\
    N_0 &= \{even(z)\} \\
    \Rightarrow M_1 &= \{even(s(0)), even(s(s(x))), even(p(0)), even(p(p(x)))\} \\
    N_1 &= \{even(s(z)), even(p(z))\} \\
    \Rightarrow M_2 &= \{even(s(s(x))), even(p(p(x)))\} \\
    N_2 &= \{even(s(s(z))), even(s(p(z))), even(p(s(z))), even(p(p(z)))\} \\
    \Rightarrow M_3 &= \varnothing \\
    N_3 &= \{even(s(p(z))), even(p(s(z)))\}
\end{align*}

Note the fact that $M_0$ does not contain left-hand sides $s(p(x))$ and $p(s(x))$ since the top-level function symbol is not even.

In the first step we unify via $\sigma = \{z \mapsto 0\}$, then apply $C(\sigma)$, i.e.: $\{z \mapsto s(z)\}$ and $\{z \mapsto p(z)\}$. We apply the same substitutions again to arrive at $M_2$. There, terms $even(s(s(x)))$ and $even(p(p(x)))$ are trivially matched by $even(s(s(z)))$ and $even(p(p(z)))$, therefore these pairs are removed. Finally, $M_3$ is empty so the algorithm stops. $N_3$ contains the patterns that are not defined for $even$, however, both these terms are further $\mathcal{R}_\mathbb{Z}$-reducible by the rules. Therefore, the definition of $even$ is relatively complete. We can notice that the algorithm indirectly also proven pattern incompleteness, as $N_3$ contains patterns where the function $even$ needs to be defined.

\end{example}

\section{Tree automata-based algorithms}
Pattern completeness of left-linear systems can also be verified using tree automata based solution, e.g. with the framework developed by Middeldorp et al. in \cite{middeldorp} or by Bouhoula and Jacquemard in \cite{bouhoula}. The experiments done by Thiemann and Yamada in \cite{thiemann} construct tree automata $\mathcal{A}$ and $\mathcal{B}$ for their test cases and verify the language inclusion problem $\mathcal{L}(\mathcal{A}) \subseteq \mathcal{L}(\mathcal{B})$ via the framework. 

A tree automaton $\mathcal{A}$ over an alphabet $\mathcal{F}$ is defined as the 4-tuple $(Q,\mathcal{F},Q_f,\Delta)$, where $Q$ is the set of states, $Q_f \subseteq Q$ are the final states, $\Delta$ are the transition rules between states. Transition rules are defined as the set of rules of the form $f(q_1(x_1), ..., q_n(x_n)) \rightarrow q(f(x_1, ..., x_n))$, where $n \geq 0$, $f \in \mathcal{F}_n$, $q, q_1, ..., q_n \in Q$, $x_1, ..., x_n \in \mathcal{X}$. A term is accepted by $\mathcal{A}$ if $t \xrightarrow[\mathcal{A}]{*} q(t), q \in Q_f$. Bottom-up tree automata start their computation at the leaves of the tree and move upwards, in contrast with top-down tree automata which start at the root. The language $\mathcal{L}(\mathcal{A})$ of tree automaton $\mathcal{A}$ is defined as the set of ground terms accepted by $\mathcal{A}$ \cite{tata}.

To translate pattern problems to tree automata domain, the following construction can be used, as demonstrated in the paper of Thiemann and Yamada\cite{thiemann}. Firstly, for the term rewrite system two tree automata $\mathcal{A}$ and $\mathcal{B}$ are constructed. Tree automata $\mathcal{A}$ accepts each valid input of the term rewrite system, whereas tree automata $\mathcal{B}$ accepts each left hand side of the system. The pattern problem then reduces to the language inclusion problem $\mathcal{L}(\mathcal{A}) \subseteq \mathcal{L}(\mathcal{B})$. Namely, that for each term recognised by tree automaton $\mathcal{A}$, there exist a matching term recognised by tree automaton $\mathcal{B}$. Conversely, if there is an input term recognised by tree automaton $\mathcal{A}$, but not recognised by tree automaton $\mathcal{B}$, then there is an incomplete pattern.

The framework by Middeldorp et al. \cite{middeldorp} constructs bottom-up tree automata to verify properties thereof, whereas the algorithm by Bouhoula et al. \cite{bouhoula} construct many-rooted top-down tree automata.

\section{Further notable work} \label{notable-work}
In \cite{thiel}, Thiel introduces calculus of components, on which the paper by Lazrek et al. is based. The complement of a term $t$ in $\mathcal{T}(\mathcal{C}, \mathcal{X})$ is defined as the finite set $T \subseteq \mathcal{T}(\mathcal{C}, \mathcal{X})$ such that $G(t) \cup G(T) = \mathcal{T}(\mathcal{C})$, i.e. the union ground terms of $t$ and $T$ equal the constructor ground terms. Their algorithm details a way to decide sufficient completeness, similar to the complement algorithm of \cite{lazrek}.

Decidability of quasi-reducibility was shown by Kapur et al. in \cite{kapur}. Their algorithm, however, is impractical in practice, as it has exponential best-case complexity. The \textit{complement algorithm} by Lazrek et al. is a refinement of this paper.

In \cite{aoto}, Aoto and Toyama introduce \textit{strong quasi-reducibility}, in their paper Ground Confluence Prover based on Rewriting Induction. Strong quasi-reducibility extends quasi-reducibility to term rewriting systems with non-free constructors, i.e. constructors that can be further reduced in the system.

Cynthia Kop presented and algorithm to decide quasi-reducibility in logically constrained term rewrite systems in \cite{cynthia}. These LCTRSs are of the nature e.g. "rule $x \rightarrow y$ is applicable only if $x > 5$".

Bouhoula and Jacquemard constructed a tree-automata based framework to decide sufficient completeness of logically constrained term rewrite systems in \cite{bouhoula}.

The Glasgow Haskell Compiler \cite{ghc} performs pattern completeness checks by enabling \texttt{-Wincomplete-patterns}. It applies, however, only to linear patterns, as non-linear patterns like \texttt{f a a = ...} are not allowed by the language, they need to be simulated by guards like \texttt{f x y | x == y = ...}.

