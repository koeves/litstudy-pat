\begin{frame}[fragile]{Functional Programming Example I/III}
Functional programs are written using pattern matching.

Critical to ensure that these patterns are complete, otherwise:

\begin{itemize}
    \item Runtime errors
    \item Untimely termination
\end{itemize}

Example:
\begin{lstlisting}
first :: [Int] -> Maybe Int
first (x:_) = Just x
\end{lstlisting}

Notice the missing case for \texttt{[]}.
\end{frame}

\begin{frame}[fragile]{Functional Programming Example II/III}
Now when we run this:

\begin{lstlisting}
main :: IO ()
main = do
    print $ first [1,2,3]
    print $ first []
\end{lstlisting}

Our program crashes:

\begin{footnotesize}
\begin{semiverbatim}
\$ ./incomplete-pattern 
Just 1
incomplete-pattern: incomplete-pattern.hs:2:1-21: 
    Non-exhaustive patterns in function first
\end{semiverbatim}
\end{footnotesize}
\end{frame}

\begin{frame}[fragile]{Functional Programming Example III/III}
We can use GHC to identify these cases:

\begin{footnotesize}
\begin{semiverbatim}
incomplete-pattern.hs:2:1: \textcolor{Purple}{warning:} [\textcolor{Purple}{-Wincomplete-patterns}]
    Pattern match(es) are non-exhaustive
    In an equation for ‘first’:
        Patterns of type ‘[Int]’ not matched: []
  |
2 | \textcolor{Purple}{first (x:xs) = Just x}
  | \textcolor{Purple}{^^^^^^^^^^^^^^^^^^^^^}
\end{semiverbatim}
\end{footnotesize}

How do we do this algorithmically?
\end{frame}

\begin{frame}{Term Rewriting I/II}
Idea:
    \begin{itemize}
        \item Analyse programs as term-rewrite systems
        \item Goal: decide when a function $f$ (like our function \texttt{first}) is \textit{pattern complete}
    \end{itemize}

Ingredients of term rewrite systems:

\begin{itemize}
    \item set of function symbols $\Sigma$ with arity $\#$
    \item set of rewrite rules $\ell \rightarrow r$
    \item set variables $\mathcal{X}$
    \item terms $\mathcal{T}(\Sigma, \mathcal{X})$ from function symbols and variables
\end{itemize}
\end{frame}

\begin{frame}{Term Rewriting II/II}
Example term rewrite system $\mathcal{R}$:

\begin{itemize}
    \item Function symbols
    \begin{itemize}
        \item Constructors: true, false, 0 (constants), s (unary)
        \item Defined symbol: even (unary)
    \end{itemize}
    \item Rewrite rules:
    \begin{itemize}
        \item $\text{even}(0) \rightarrow \text{true}$
        \item $\text{even}(s(0)) \rightarrow \text{false}$ 
        \item $\text{even}(s(s(x))) \rightarrow \text{even(x)}$
    \end{itemize}
\end{itemize}

\begin{block}{Example – Reduction}
$\text{even}(\textcolor{red}{s(s(}s(0)\textcolor{red}{))}) \rightarrow \text{even}(\textcolor{red}{s(0)}) \rightarrow \text{false}$ \\

$\text{even}(\textcolor{red}{s(s(}s(s(0))\textcolor{red}{))}) \rightarrow \text{even}(\textcolor{red}{s(s(}0\textcolor{red}{))}) \rightarrow \text{even}(\textcolor{red}{0}) \rightarrow \text{true}$
\end{block}


\end{frame}

% \begin{frame}{Preliminaries \& Notation}
% \begin{itemize}
%     \item Signature $\Sigma$ of function symbols and set $\mathcal{X}$ of variables
%     \item Terms $\mathcal{T}(\Sigma, \mathcal{X})$ are the smallest set such that:
%         \begin{itemize}
%             \item $\mathcal{X} \subseteq \mathcal{T}(\Sigma, \mathcal{X})$
%             \item $f(t_1, ..., t_n) \in \mathcal{T}(\Sigma, \mathcal{X})$, $f \in \Sigma$ of arity $n$, $t_1, ..., t_n \in \mathcal{X}$
%         \end{itemize}
%     \item Signature $\Sigma$ consists of constructors $\mathcal{C}$ and defined symbols $\mathcal{D}$
%     \item Ground terms $\mathcal{T}(\Sigma)$ and ground constructor terms $\mathcal{T}(\mathcal{C})$ are terms without variables
%     \item A substitution $\sigma$ is a mapping $\mathcal{X} \rightarrow \mathcal{T}(\Sigma, \mathcal{X})$
%     \item Set $R$ contains rewrite rules $(\ell \rightarrow r)$ between $\mathcal{T}(\Sigma, \mathcal{X})$ s.t. $\ell \notin \mathcal{X}$ and $\mathcal{V}ar(r) \subseteq \mathcal{V}ar(\ell)$
%     \item Term rewrite system $\mathcal{R}$ is a pair $(\Sigma, R)$
% \end{itemize}
% \end{frame}

\begin{frame}{Matching I/II}
\begin{block}{Definition}
    \textit{Matching problem}: given terms $s$ and $t$, find substitution $\sigma$ from $\mathcal{X}$ to $\mathcal{T}(\Sigma, \mathcal{X})$ such that $s\sigma = t$.
\end{block}

\begin{block}{Example}
\begin{itemize}
    \item Match $z$ to $0$. Take $\sigma = \{z \mapsto 0\}$
    \item Match $\text{even}(z)$ to $0$. No such $\sigma$ exists
    \item Match $f(a, b)$ to $f(x, x)$. Take $\sigma = \{a \mapsto x, b \mapsto x\}$
    \item Match $f(a, a)$ to $f(x, s(x))$. No such $\sigma$ exists
\end{itemize}

\end{block}
\end{frame}

\begin{frame}{Matching II/II}
Idea:
\begin{itemize}
    \item Represent defined function $f$ as TRS
    \item Match input term $f(z_1, ..., z_n)$ to LHS of TRS (domain of $f$) with $z_i$ some constructor term
    \item If for all constructor substitution for $z_i$ we find a match: $f$ is pattern complete
\end{itemize}

\begin{block}{Example}
Previous TRS with defined symbol even. Matching problems:
\begin{itemize}
    \item even(z) to even(0)
    \item even(z) to even(s(0))
    \item even(z) to even(s(s(x)))
\end{itemize}
\end{block}
\end{frame}

% Matching problem $mp$ can be represented as a set of pairs of terms:
% $$mp = \{(t_1, \ell_1), ..., (t_n, \ell_n)\} \subseteq \mathcal{T}(\mathcal{F}, \mathcal{X}) \times \mathcal{T}(\mathcal{F}, \mathcal{X})$$

% \textit{Pattern problem} $pp$ is a finite set of matching problems.


% \begin{frame}{Towards Pattern Completeness II.}
% \begin{block}{Definition – Completeness}
% \begin{itemize}
%     \item A matching problem is \textit{complete}: if given substitution $\sigma : \mathcal{X} \mapsto \mathcal{T}(\mathcal{C})$, $\forall (t, \ell) \in mp.\ \exists \gamma : \mathcal{X} \mapsto \mathcal{T}(\Sigma).\ t\sigma = \ell\gamma$. 
     
%      \item A pattern problem is complete if for each constructor ground substitution $\sigma$ there is some $mp \in pp$ that is complete. 
     
%      \item A set $P$ of pattern problems is complete if all $pp \in P$ are complete. 
% \end{itemize}
% \end{block}
% \end{frame}

% \begin{frame}{Example – Matching problem}
% Given TRS $\mathcal{R}_\mathbb{N}$ from \cite{thiemann}:
% \begin{align*}
% \mathcal{C} &= \{\text{true}: \mathbb{B}, \text{false}: \mathbb{B}, 0: \mathbb{N}, s: \mathbb{N} \rightarrow \mathbb{N}\} \\
% \mathcal{D} &= \{\text{even}: \mathbb{N} \rightarrow \mathbb{B}\} \\
% R &= \{\text{even}(0) \rightarrow \text{true},\ \text{even}(s(0)) \rightarrow \text{false},\ \text{even}(s(s(x))) \rightarrow \text{true}\}
% \end{align*}

% Consider the following matching problems:
% \begin{align*}
%     mp_1 &= \{(z, 0)\} & mp_2 &= \{(\text{even}(z), 0)\}
% \end{align*}

% \begin{itemize}
%     \item $mp_1$ is complete with respect to $\sigma = \{z \mapsto 0\}$
%     \item $mp_2$ is incomplete since $\nexists \sigma.\ \text{even}(x)^\sigma = 0$
% \end{itemize}
%  % is incomplete since there exists no $\sigma$ such that $$. The set of pattern problems describing this program would be: $$P = \{\{\{(\text{even}(z), \text{even}(0))\}, \{(\text{even}(z), \text{even}(s(x)))\}, \{(\text{even}(z), \text{even}(s(s(x))))\}\}\}$$
% \end{frame}

% \begin{frame}{Pattern Completeness}

% \begin{block}{Definition}
%     A term rewrite system $\mathcal{R}$ with left hand sides $L$ is \textit{pattern complete} if every ground term $t = f(x_1, ..., x_n)$ is matched by some $\ell \in L$.
% \end{block}

% The question whether a program with left hand sides $L$ and defined symbols $\mathcal{D}$ is pattern complete can be expressed with the following set of pattern problems\cite{thiemann}:
% $$P = \{\{\{(f(x_1, ..., x_n), \ell)\} \mid \ell \in L\} \mid f \in \mathcal{D}\}$$

% \end{frame}

% \begin{frame}{Example – Pattern completeness}
% Given TRS $\mathcal{R}_\mathbb{N}$ from \cite{thiemann}:
% \begin{align*}
% \mathcal{C} &= \{\text{true}: \mathbb{B}, \text{false}: \mathbb{B}, 0: \mathbb{N}, s: \mathbb{N} \rightarrow \mathbb{N}\} \\
% \mathcal{D} &= \{\text{even}: \mathbb{N} \rightarrow \mathbb{B}\} \\
% R &= \{\text{even}(0) \rightarrow \text{true},\ \text{even}(s(0)) \rightarrow \text{false},\ \text{even}(s(s(x))) \rightarrow \text{true}\}
% \end{align*}

% The set of pattern problems describing this program would be: 
% \begin{align*}
%     P = \{\{&\{(\text{even}(z), \text{even}(0))\}, \\
%     &\{(\text{even}(z), \text{even}(s(x)))\},\\ 
%     &\{(\text{even}(z), \text{even}(s(s(x))))\}\}\}
% \end{align*}
% \end{frame}

% \begin{frame}{Quasi-reducibility}

% \begin{block}{Definition}
%     A term rewrite system $\mathcal{R}$ with left hand sides $L$ is \textit{quasi-reducible} if every ground term $t = f(x_1, ..., x_n)$ has a subterm $t'\trianglelefteq t$ that is matched by some $\ell \in L$.
% \end{block}

% Notice how quasi-reducibility is more permissive, it allows for matching under the root!
% \end{frame}