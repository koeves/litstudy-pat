\documentclass[onlytextwidth]{beamer}
\usepackage[utf8]{inputenc}
\usepackage{listings}
\usepackage{graphicx}
\usepackage{amssymb}
\usepackage{amsmath}
\usepackage[dvipsnames]{xcolor}
\usetheme{Berlin}

\usepackage[sorting=none]{biblatex}
\renewbibmacro{in:}{%
  \ifentrytype{article}{}{\printtext{\bibstring{in}\intitlepunct}}}
\addbibresource{cite.bib}

\usepackage{listings}
\usepackage{xcolor}

\definecolor{keywordcolor}{rgb}{0.1, 0.1, 0.6}
\definecolor{stringcolor}{rgb}{0.63, 0.125, 0.94}
\definecolor{commentcolor}{rgb}{0.3, 0.4, 0.4}
\definecolor{identifiercolor}{rgb}{0.0, 0.2, 0.4}
\definecolor{backgroundcolor}{rgb}{0.97, 0.97, 0.97}

\lstset{
  language=Haskell,
  backgroundcolor=\color{backgroundcolor},
  basicstyle=\ttfamily\small,
  keywordstyle=\color{keywordcolor}\bfseries,
  stringstyle=\color{stringcolor},
  commentstyle=\color{commentcolor}\itshape,
  identifierstyle=\color{identifiercolor},
  numbers=left,
  numberstyle=\tiny\color{gray},
  stepnumber=1,
  numbersep=10pt,
  showstringspaces=false,
  tabsize=2,
  breaklines=true,
  breakatwhitespace=false,
  morekeywords={data, type, newtype, deriving, where, let, in, do, module, import},
  frame=single,
  framesep=8pt,
  xleftmargin=8pt,
  xrightmargin=8pt
}

\graphicspath{{images/}}

\AtBeginSection[]
{
  \begin{frame}
    \frametitle{Outline}
    \tableofcontents[currentsection]
  \end{frame}
}

\usenavigationsymbolstemplate{}

\setbeamercovered{transparent}

\title{Literature Study on Algorithms \\for Pattern Completeness}
\author[Daniel Köves]{
    Daniel Köves\\
    \vspace*{0.3cm}
    \scriptsize{Femke van Raamsdonk}\\
    \scriptsize{Jörg Endrullis}
}
\institute{Vrije Universiteit Amsterdam}
\date{November 2024}
\logo{\includegraphics[height=1cm]{vu-griffioen.pdf}}

\begin{document}

\frame{\titlepage}

\section{Introduction}
\begin{frame}[fragile]{Functional Programming Example I/III}
Functional programs are written using pattern matching.

Critical to ensure that these patterns are complete, otherwise:

\begin{itemize}
    \item Runtime errors
    \item Untimely termination
\end{itemize}

Example:
\begin{lstlisting}
first :: [Int] -> Maybe Int
first (x:_) = Just x
\end{lstlisting}

Notice the missing case for \texttt{[]}.
\end{frame}

\begin{frame}[fragile]{Functional Programming Example II/III}
Now when we run this:

\begin{lstlisting}
main :: IO ()
main = do
    print $ first [1,2,3]
    print $ first []
\end{lstlisting}

Our program crashes:

\begin{footnotesize}
\begin{semiverbatim}
\$ ./incomplete-pattern 
Just 1
incomplete-pattern: incomplete-pattern.hs:2:1-21: 
    Non-exhaustive patterns in function first
\end{semiverbatim}
\end{footnotesize}
\end{frame}

\begin{frame}[fragile]{Functional Programming Example III/III}
We can use GHC to identify these cases:

\begin{footnotesize}
\begin{semiverbatim}
incomplete-pattern.hs:2:1: \textcolor{Purple}{warning:} [\textcolor{Purple}{-Wincomplete-patterns}]
    Pattern match(es) are non-exhaustive
    In an equation for ‘first’:
        Patterns of type ‘[Int]’ not matched: []
  |
2 | \textcolor{Purple}{first (x:xs) = Just x}
  | \textcolor{Purple}{^^^^^^^^^^^^^^^^^^^^^}
\end{semiverbatim}
\end{footnotesize}

How do we do this algorithmically?
\end{frame}

\begin{frame}{Term Rewriting I/II}
Idea:
    \begin{itemize}
        \item Analyse programs as term-rewrite systems
        \item Goal: decide when a function $f$ (like our function \texttt{first}) is \textit{pattern complete}
    \end{itemize}

Ingredients of term rewrite systems:

\begin{itemize}
    \item set of function symbols $\Sigma$ with arity $\#$
    \item set of rewrite rules $\ell \rightarrow r$
    \item set variables $\mathcal{X}$
    \item terms $\mathcal{T}(\Sigma, \mathcal{X})$ from function symbols and variables
\end{itemize}
\end{frame}

\begin{frame}{Term Rewriting II/II}
Example term rewrite system $\mathcal{R}$:

\begin{itemize}
    \item Function symbols
    \begin{itemize}
        \item Constructors: true, false, 0 (constants), s (unary)
        \item Defined symbol: even (unary)
    \end{itemize}
    \item Rewrite rules:
    \begin{itemize}
        \item $\text{even}(0) \rightarrow \text{true}$
        \item $\text{even}(s(0)) \rightarrow \text{false}$ 
        \item $\text{even}(s(s(x))) \rightarrow \text{even(x)}$
    \end{itemize}
\end{itemize}

\begin{block}{Example – Reduction}
$\text{even}(\textcolor{red}{s(s(}s(0)\textcolor{red}{))}) \rightarrow \text{even}(\textcolor{red}{s(0)}) \rightarrow \text{false}$ \\

$\text{even}(\textcolor{red}{s(s(}s(s(0))\textcolor{red}{))}) \rightarrow \text{even}(\textcolor{red}{s(s(}0\textcolor{red}{))}) \rightarrow \text{even}(\textcolor{red}{0}) \rightarrow \text{true}$
\end{block}


\end{frame}

% \begin{frame}{Preliminaries \& Notation}
% \begin{itemize}
%     \item Signature $\Sigma$ of function symbols and set $\mathcal{X}$ of variables
%     \item Terms $\mathcal{T}(\Sigma, \mathcal{X})$ are the smallest set such that:
%         \begin{itemize}
%             \item $\mathcal{X} \subseteq \mathcal{T}(\Sigma, \mathcal{X})$
%             \item $f(t_1, ..., t_n) \in \mathcal{T}(\Sigma, \mathcal{X})$, $f \in \Sigma$ of arity $n$, $t_1, ..., t_n \in \mathcal{X}$
%         \end{itemize}
%     \item Signature $\Sigma$ consists of constructors $\mathcal{C}$ and defined symbols $\mathcal{D}$
%     \item Ground terms $\mathcal{T}(\Sigma)$ and ground constructor terms $\mathcal{T}(\mathcal{C})$ are terms without variables
%     \item A substitution $\sigma$ is a mapping $\mathcal{X} \rightarrow \mathcal{T}(\Sigma, \mathcal{X})$
%     \item Set $R$ contains rewrite rules $(\ell \rightarrow r)$ between $\mathcal{T}(\Sigma, \mathcal{X})$ s.t. $\ell \notin \mathcal{X}$ and $\mathcal{V}ar(r) \subseteq \mathcal{V}ar(\ell)$
%     \item Term rewrite system $\mathcal{R}$ is a pair $(\Sigma, R)$
% \end{itemize}
% \end{frame}

\begin{frame}{Matching I/II}
\begin{block}{Definition}
    \textit{Matching problem}: given terms $s$ and $t$, find substitution $\sigma$ from $\mathcal{X}$ to $\mathcal{T}(\Sigma, \mathcal{X})$ such that $s\sigma = t$.
\end{block}

\begin{block}{Example}
\begin{itemize}
    \item Match $z$ to $0$. Take $\sigma = \{z \mapsto 0\}$
    \item Match $\text{even}(z)$ to $0$. No such $\sigma$ exists
    \item Match $f(a, b)$ to $f(x, x)$. Take $\sigma = \{a \mapsto x, b \mapsto x\}$
    \item Match $f(a, a)$ to $f(x, s(x))$. No such $\sigma$ exists
\end{itemize}

\end{block}
\end{frame}

\begin{frame}{Matching II/II}
Idea:
\begin{itemize}
    \item Represent defined function $f$ as TRS
    \item Match input term $f(z_1, ..., z_n)$ to LHS of TRS (domain of $f$) with $z_i$ some constructor term
    \item If for all constructor substitution for $z_i$ we find a match: $f$ is pattern complete
\end{itemize}

\begin{block}{Example}
Previous TRS with defined symbol even. Matching problems:
\begin{itemize}
    \item even(z) to even(0)
    \item even(z) to even(s(0))
    \item even(z) to even(s(s(x)))
\end{itemize}
\end{block}
\end{frame}

% Matching problem $mp$ can be represented as a set of pairs of terms:
% $$mp = \{(t_1, \ell_1), ..., (t_n, \ell_n)\} \subseteq \mathcal{T}(\mathcal{F}, \mathcal{X}) \times \mathcal{T}(\mathcal{F}, \mathcal{X})$$

% \textit{Pattern problem} $pp$ is a finite set of matching problems.


% \begin{frame}{Towards Pattern Completeness II.}
% \begin{block}{Definition – Completeness}
% \begin{itemize}
%     \item A matching problem is \textit{complete}: if given substitution $\sigma : \mathcal{X} \mapsto \mathcal{T}(\mathcal{C})$, $\forall (t, \ell) \in mp.\ \exists \gamma : \mathcal{X} \mapsto \mathcal{T}(\Sigma).\ t\sigma = \ell\gamma$. 
     
%      \item A pattern problem is complete if for each constructor ground substitution $\sigma$ there is some $mp \in pp$ that is complete. 
     
%      \item A set $P$ of pattern problems is complete if all $pp \in P$ are complete. 
% \end{itemize}
% \end{block}
% \end{frame}

% \begin{frame}{Example – Matching problem}
% Given TRS $\mathcal{R}_\mathbb{N}$ from \cite{thiemann}:
% \begin{align*}
% \mathcal{C} &= \{\text{true}: \mathbb{B}, \text{false}: \mathbb{B}, 0: \mathbb{N}, s: \mathbb{N} \rightarrow \mathbb{N}\} \\
% \mathcal{D} &= \{\text{even}: \mathbb{N} \rightarrow \mathbb{B}\} \\
% R &= \{\text{even}(0) \rightarrow \text{true},\ \text{even}(s(0)) \rightarrow \text{false},\ \text{even}(s(s(x))) \rightarrow \text{true}\}
% \end{align*}

% Consider the following matching problems:
% \begin{align*}
%     mp_1 &= \{(z, 0)\} & mp_2 &= \{(\text{even}(z), 0)\}
% \end{align*}

% \begin{itemize}
%     \item $mp_1$ is complete with respect to $\sigma = \{z \mapsto 0\}$
%     \item $mp_2$ is incomplete since $\nexists \sigma.\ \text{even}(x)^\sigma = 0$
% \end{itemize}
%  % is incomplete since there exists no $\sigma$ such that $$. The set of pattern problems describing this program would be: $$P = \{\{\{(\text{even}(z), \text{even}(0))\}, \{(\text{even}(z), \text{even}(s(x)))\}, \{(\text{even}(z), \text{even}(s(s(x))))\}\}\}$$
% \end{frame}

% \begin{frame}{Pattern Completeness}

% \begin{block}{Definition}
%     A term rewrite system $\mathcal{R}$ with left hand sides $L$ is \textit{pattern complete} if every ground term $t = f(x_1, ..., x_n)$ is matched by some $\ell \in L$.
% \end{block}

% The question whether a program with left hand sides $L$ and defined symbols $\mathcal{D}$ is pattern complete can be expressed with the following set of pattern problems\cite{thiemann}:
% $$P = \{\{\{(f(x_1, ..., x_n), \ell)\} \mid \ell \in L\} \mid f \in \mathcal{D}\}$$

% \end{frame}

% \begin{frame}{Example – Pattern completeness}
% Given TRS $\mathcal{R}_\mathbb{N}$ from \cite{thiemann}:
% \begin{align*}
% \mathcal{C} &= \{\text{true}: \mathbb{B}, \text{false}: \mathbb{B}, 0: \mathbb{N}, s: \mathbb{N} \rightarrow \mathbb{N}\} \\
% \mathcal{D} &= \{\text{even}: \mathbb{N} \rightarrow \mathbb{B}\} \\
% R &= \{\text{even}(0) \rightarrow \text{true},\ \text{even}(s(0)) \rightarrow \text{false},\ \text{even}(s(s(x))) \rightarrow \text{true}\}
% \end{align*}

% The set of pattern problems describing this program would be: 
% \begin{align*}
%     P = \{\{&\{(\text{even}(z), \text{even}(0))\}, \\
%     &\{(\text{even}(z), \text{even}(s(x)))\},\\ 
%     &\{(\text{even}(z), \text{even}(s(s(x))))\}\}\}
% \end{align*}
% \end{frame}

% \begin{frame}{Quasi-reducibility}

% \begin{block}{Definition}
%     A term rewrite system $\mathcal{R}$ with left hand sides $L$ is \textit{quasi-reducible} if every ground term $t = f(x_1, ..., x_n)$ has a subterm $t'\trianglelefteq t$ that is matched by some $\ell \in L$.
% \end{block}

% Notice how quasi-reducibility is more permissive, it allows for matching under the root!
% \end{frame}

\section{Thiemann and Yamada's algorithm}
\begin{frame}{Properties}
\begin{itemize}
    \item Algorithm to decide pattern completeness
    \item \textbf{Input}: matching problems $(f(z), \ell)$
    \begin{itemize}
        \item For each LHS $\ell$ of TRS from $f$
        \item $z$ some arbitrary constructor term
    \end{itemize}
    \item \textbf{Output}: success or failure
    \item Flow: 
    \begin{itemize}
        \item Iteratively decompose terms until match/clash
        \item If we encounter $z$ with $\ell$ not a variable, instantiate $z$ via $\sigma = \{z \mapsto c(x_1, ..., x_n) \}$ for each constructor $c$
    \end{itemize}
\end{itemize}


\end{frame}

\begin{frame}{Example I}
Constructors: $0$, $s$, function $f$: $f(\textcolor{red}{0}) \rightarrow 0$ and $f(\textcolor{red}{s}(x)) \rightarrow x$

\begin{block}{Algorithm}
\begin{columns}[t]

\begin{column}{.5\textwidth}
\centering{
    $(\textcolor{red}{f}(z), \textcolor{red}{f}(0))$\\
    \onslide<2->{$(\textcolor{red}{z}, 0)$\\}

    \vspace{0.5em}
    \onslide<2->{\hrule{}}
    \vspace{0.5em}

    \begin{columns}[t]

    \begin{column}{.5\textwidth}
    \centering{
        \onslide<3->{$(\textcolor{red}{0}, \textcolor{red}{0})$\\
\textcolor{Green}{match}\\}
    }
    \end{column}
    
    \onslide<3->{\vrule{}}
    
    \begin{column}{.5\textwidth}
    \centering{
        \onslide<3->{$(\textcolor{red}{0}, \textcolor{red}{s}(x))$\\
clash\\}
    }
    \end{column}
    \end{columns}
}
\end{column}
\vrule{}

\begin{column}{.5\textwidth}
\centering{
    $(\textcolor{red}{f}(z), \textcolor{red}{f}(s(x)))$\\
    \onslide<2->{$(\textcolor{red}{z}, s(x))$\\}

    \vspace{0.5em}
    \onslide<2->{\hrule{}}
    \vspace{0.5em}

    \begin{columns}[t]
    
    \begin{column}{.5\textwidth}
    \centering{
        \onslide<3->{$(\textcolor{red}{s}(z), \textcolor{red}{0})$\\
clash\\}
    }
    \end{column}
    
    \onslide<3->{\vrule{}}
    
    \begin{column}{.5\textwidth}
    \centering{
        \onslide<3->{$(\textcolor{red}{s}(z), \textcolor{red}{s}(x))$\\
$(\textcolor{red}{z}, \textcolor{red}{x})$\\
        $\text{\textcolor{Green}{match}}^{\tiny{\sigma=\{x \mapsto z\}}}$}
    }
    \end{column}
    
    \end{columns}
} 
\end{column}
\end{columns}
\end{block}

\onslide<4->{Result: \textbf{success}, since each constructor substitution for $z$ results in \textcolor{Green}{match}}

\end{frame}

\begin{frame}{Example II – Incomplete pattern}
Constructors: $0$, $s$, function $f$: $f(\textcolor{red}{0}) \rightarrow 0$ (missing case for s(x)).

\begin{block}{Algorithm}
\centering{
$(\textcolor{red}{f}(z), \textcolor{red}{f}(0))$ \\
\onslide<2->{$(\textcolor{red}{z}, \textcolor{red}{0})$ \\}
}

\vspace{0.5em}
\onslide<3->{\hrule{}}
\vspace{0.5em}

\begin{columns}[t]

\begin{column}{.5\textwidth}
\centering{
\onslide<3->{$(\textcolor{red}{0}, \textcolor{red}{0})$ \\
\textcolor{Green}{match}\\}
}
\end{column}

\onslide<3->{\vrule{}}

\begin{column}{.5\textwidth}
\centering{
\onslide<3->{$(\textcolor{red}{s}(z), \textcolor{red}{0})$ \\
clash\\}
}
\end{column}

\end{columns}
\end{block}

\onslide<4->{Result: \textbf{failure}, since there's no \textcolor{Green}{match}
for substitution $\sigma = \{z \mapsto s(z)\}$}
\end{frame}

\begin{frame}{Example III – General case}
Function $f$: with LHS $f(x, x)$ and $f(x, y)$\\

Pattern $f(x, x)$ is called \textit{non-linear}, due to the repeated variable $x$

\begin{block}{Algorithm}
\begin{columns}[t] 

\begin{column}{.5\textwidth}
\centering{
    $(\textcolor{red}{f}(a, b), \textcolor{red}{f}(x, x))$\\
    \onslide<2->{$(a, \textcolor{red}{x}), (b, \textcolor{red}{x})$\\}
    \onslide<3->{clash}
}
\end{column}
\onslide<3->{\vrule{}}

\begin{column}{.5\textwidth}
\centering{
    $(\textcolor{red}{f}(a, b), \textcolor{red}{f}(x, y))$\\
    \onslide<2->{$(\textcolor{red}{a}, x), (\textcolor{red}{b}, y)$\\}
    \onslide<3->{$\text{\textcolor{Green}{match}}^{\ \footnotesize \sigma_1 = \{x \mapsto a\}\ \sigma_2 = \{y \mapsto b\}}$}
}
\end{column}
\end{columns}
\end{block}

\onslide<4->{
\begin{itemize}
    \item Result: \textbf{success}, since right-side matches both $a$ and $b$ (which are arbitrary constructor terms)
    \item Left side results in clash since we cannot match variable x to both $a$ and $b$
\end{itemize}
}

% \begin{footnotesize}
% \begin{align*}
% \onslide<2->{P &= \{\{\{(\textcolor{red}{f}(a, b), \textcolor{red}{f}(x, x))\}, \{(\textcolor{red}{f}(a, b), \textcolor{red}{f}(x, y))\}\}}\\
% \onslide<3->{\text{decompose} &\Rrightarrow^{*} \{\{\{(a, \textcolor{red}{x}), (b, \textcolor{red}{x})\}, \{(a, x), (b, y)\}\}\}} \\
% \onslide<4->{\textcolor{red}{clash'} &\Rrightarrow \{\{\textcolor{red}{\bot_{mp}}, \{(a, x), (b, y)\}\}\}} \\
% \onslide<5->{\text{remove-mp} &\Rrightarrow \{\{\{(a, \textcolor{red}{x}), (b, \textcolor{red}{y})\}\}\}} \\
% \onslide<6->{\text{match} &\Rrightarrow^{*} \{\textcolor{red}{\{\varnothing\}}\}} \\
% \onslide<7->{\text{success} &\Rrightarrow^{*} \{\textcolor{red}{\top_{pp}}\}} \\
% \onslide<8->{\text{remove-pp} &\Rrightarrow^{*} \varnothing}
% \end{align*}
% \end{footnotesize}

\end{frame}

% \begin{frame}{Steps}
% Rewrite matching problems: 
%     \begin{itemize}
%         \item decompose: $(\textcolor{red}{f}(t_1, ..., t_n), \textcolor{red}{f}(l_1, ..., l_n)) \rightarrow (t_1, l_1), ..., (t_n, l_n)$
%         \item match: $(\textcolor{red}{t}, \textcolor{red}{x}) \rightarrow \varnothing$
%         \item clash: $(\textcolor{red}{f}(...), \textcolor{red}{g}(...)) \rightarrow \bot$
%     \end{itemize}

% Rewrite results of the matching problem reductions with:
% \begin{itemize}
%     \item $\bot \rightarrow \varnothing$
%     \item $\varnothing \rightarrow \top$
% \end{itemize}

% Rewrite the remainder with:
% \begin{itemize}
%     \item failure: $\varnothing \rightarrow \bot$
%     \item success: $\top \rightarrow \varnothing$
%     \item instantiate: $pp \rightarrow$ Inst(pp,x) if $x, f(...)) \in pp$
% \end{itemize}

% \end{frame}

% \begin{frame}{The algorithm – left linear case}
% \begin{footnotesize}
% For matching problems $mp$:
% \begin{align*}
% \textbf{decompose} & & \{(f(t_1, ..., t_n), f(l_1, ..., l_n))\} &\rightarrow \{(t_1, l_1), ..., (t_n, l_n)\} \\
% \textbf{match} & & \{(t, x)\} &\rightarrow \varnothing \text{ if } \forall (t', l) \in mp \text{. } x \notin Var(l) \\
% \textbf{clash} & & \{(f(...), g(...))\} &\rightarrow \bot_{mp}\text{ if }f \neq g
% \intertext{For pattern problems $pp$:}
% \textbf{remove-mp} & & \{\bot_{mp}\} &\rightarrow \varnothing \\
% \textbf{success} & & \{\varnothing\} &\rightarrow \top_{pp}
% \intertext{For sets of pattern problems $P$:}
% \textbf{failure} & & \{\varnothing\} &\rightarrow \bot_P \\
% \textbf{remove-pp} & & \{\top_{pp}\} &\rightarrow \varnothing \\
% \textbf{instantiate} & & \{pp\} &\rightarrow \text{Inst$(pp,x)$ if }\{(x, f(...))\} \in pp
% \end{align*}
% \end{footnotesize}
% \end{frame}

% \begin{frame}{General case}
% Following further rules are introduced:
% \begin{footnotesize}
% \begin{align*}
% \textbf{clash'} & & \{(t,x), (t',x)\} \in mp &\rightarrow \bot_{mp}\text{ if $t$ and $t'$ clash} \\
% \textbf{instantiate'} & & \{\{(t,x), (t',x)\}\} \in P &\rightarrow \text{Inst$(pp,x)$} \\
% \textbf{failure'} & & \{pp\} \in P &\rightarrow \bot_P
% \end{align*}
% \end{footnotesize}

% \begin{itemize}
%     \item We apply \textbf{instantiate'} if $t$ and $t'$ differ in variable $x$ of finite sort.
%     \item We apply \textbf{failure'}, if each $mp \in pp$ there exists $\{(t,x),(t',x)\}$ such that $t$ and $t'$ differ in variable $x$ of infinite sort. 
%     \begin{itemize}
%         \item We cannot instantiate a variable of infinite sort
%         \item If not all matching problems differ in such a variable, we can mark those problems locally as $\bot_{mp}$ (via \textbf{clash'}).
%     \end{itemize}
% \end{itemize}
% \end{frame}



\section{The complement algorithm}
\begin{frame}{Properties}
\begin{itemize}
    \item Due to Lazrek et al.
    \item Can be used to decide pattern completeness
    \item \textbf{Input}: TRS $\mathcal{R}$ and defined symbol $f$
    \item \textbf{Output}: Set of constructors where $f$ is not defined    
\end{itemize}

\end{frame}

\begin{frame}{Idea}
    \begin{itemize}
        \item Start with set $M_0$ with LHSs of function $f$, set $N_0$ $f(z_1, ..., z_n)$ where $z_i$ is some constructor term 
        \item Try to unify elements of $m \in M_i$ and $n \in N_i$ with substitution $\sigma$
        \item Compute \textit{complement} of this substitution $\rho$
        \item Replace matched element in $N_i$ with new elements $n\rho$
        \item Repeat until either $M_{last}$ or $N_{last}$ is empty, or no further unification is possible
        \item If $M_{last}$ is empty but $N_{last}$ is not empty, $f$ is not defined on the terms in $N_{last}$
    \end{itemize}
\end{frame}

% \begin{frame}{Definitions}
% \begin{block}{Definition}
% The \textit{complement} of a ground term $t = c_j(t_1, ... , t_n)$ given $\mathcal{C} = \{c_1, ..., c_n\}$, $1 \leq j \leq n$ is the set $C(t)$:
% \begin{itemize}
%     \item The complement of $x \in \mathcal{X}$ is $x$
%     \item Otherwise the set $\{c_i(C(t_1), ..., C(t_n) \mid 1\leq i \leq n, i \neq j\}$
% \end{itemize}
% \end{block}

% \begin{block}{Definition}
%     The \textit{complement} of a substitution $\sigma$ is the set $C(\sigma)$ of  substitutions $\rho \neq \sigma$, with $\mathcal{D}om(\sigma) = \mathcal{D}om(\rho)$ mapping elements to complementary terms.
% \end{block}

% \end{frame}

% \begin{frame}{The algorithm}
% Initially set $M_0$ as the set of left-hand sides of $f$ and $N_0$ as the set of ground instances of $f$.
% \begin{enumerate}
%     \item Find substitution $\sigma$ to unify elements $m \in M_i$ and $n \in N_i$
%     \item Construct $M_{i+1}$ and $N_{i+1}$:
%     \begin{align*}
%     M_{i+1} &= M_{i-1} \setminus \{m\} \cup \{m\rho \mid \rho \in C(\sigma), m\rho \neq m\sigma\} \\
%     N_{i+1} &= N_{i-1} \setminus \{n\} \cup \{n\rho \mid \rho \in C(\sigma), n\rho \neq n\sigma\}
%     \end{align*}
%     \item Repeat until either $M_{last}$ or $N_{last}$ is empty, or no further unification is possible
% \end{enumerate}
% \end{frame}

% \begin{frame}{Properties}
% \begin{enumerate}
%     \item Both $M_{last}$ and $N_{last}$ are empty, the definition of $f$ is \textit{sufficiently complete}
%     \item If $M_{last}$ is empty but $N_{last}$ is not empty, $f$ is not defined on the terms in $N_{last}$
% \end{enumerate}

% Quasi-reducibility is determined when:

% \begin{itemize}
%     \item each input term is matched by the LHS of $f$ (point \textbf{1} – also pattern completeness), or
%     \item elements of $N_{last}$ can further be reduced (point \textbf{2})
% \end{itemize}
% \end{frame}

\begin{frame}{Example I - Linear case}
Constructors: $0$, $s$, function $f$: $f(\textcolor{red}{0}) \rightarrow 0$ and $f(\textcolor{red}{s}(x)) \rightarrow x$

\begin{block}{Algorithm}
\begin{center}
\begin{tabular}{ c || c | c || c | c }
    & M & N & $\sigma$ & $\rho$ \\
    \hline\hline
    0 & $f(\textcolor{red}{0}), f(s(x))$ & $f(\textcolor{red}{z})$ & $z \mapsto 0$ & $z \mapsto s(z)$ \onslide<2->\\
    1 & $f(\textcolor{red}{s}(x))$ & $f(\textcolor{red}{s}(z))$ & $z \mapsto x$ & \onslide<3->\\
    2 &  $\varnothing$ & $\varnothing$ & & \onslide<4->\\
\end{tabular}
\end{center}
\end{block}

\onslide<4->{Result: \textbf{success}, $N_2$ is empty}
\end{frame}

\begin{frame}{Example II - Incomplete pattern}
Constructors: $0$, $s$, function $f$: LHS $f(0)$ (missing case for $s(x)$).

\begin{block}{Algorithm}
\begin{center}
\begin{tabular}{ c || c | c || c | c }
    & M & N & $\sigma$ & $\rho$ \\
    \hline\hline
    0 & $f(\textcolor{red}{s}(x))$ & $f(\textcolor{red}{z})$ & $z \mapsto s(x)$ & $z \mapsto 0$ \onslide<2-> \\
    1 & $\varnothing$ & $f(0)$ & & \onslide<3->\\
\end{tabular}
\end{center}
\end{block}

\onslide<3->{Result: \textbf{failure}, $N_1$ is not empty}
\end{frame}

\section{Conclusion}
\begin{frame}{Comparison}
    \begin{itemize}
        \item Both Thiemann and Yamada's and the complement algorithm can be used to decide pattern completeness
        \item Complement algorithm's $N_{last}$ set contains patterns where $f$ still needs to be defined
        \item Thiemann and Yamada's algorithm is proven to work for non-linear patterns, whereas complement algorithm might fail
        \item Complement algorithm has built-in counterexample-generation
    \end{itemize}
\end{frame}

\begin{frame}{Conclusion}
    \begin{itemize}
        \item Literature study to compare algorithms for pattern completeness
        \item Detailed comparison between Thiemann and Yamada's algorithm and the complement algorithm of Lazrek et al.
        \item Further research could:
            \begin{itemize}
                \item Perform more thorough performance comparison between algorithms
                \item More analysis as to why Thiemann and Yamada's version outperforms the other variants
                \item Suggestion as per Thiemann and Yamada: construct a similar syntax-based algorithm to decide quasi-reducibility
            \end{itemize}
    \end{itemize}
\end{frame}

\begin{frame}{Further notable work}
    \begin{itemize}
        \item Calculus of components by Thiel on which the complement algorithm by Lazrek et al is based on
        \item Decidability of quasi-reducibility by Kapur et al.
        \item Aota and Toyama introduce \textit{strong quasi-reducibility} 
        \item Kop derives quasi-reducibility of logically constrained TRSs
        \item Bouhoula et al. use tree-automata based algorithm to decide sufficient completeness
    \end{itemize}
\end{frame}

\begin{frame}[allowframebreaks]{References}
\nocite{*}
\renewcommand*{\bibfont}{\scriptsize}
\printbibliography
\end{frame}

\end{document}