\begin{frame}{Properties}
\begin{itemize}
    \item Due to Lazrek et al.
    \item Can be used to decide pattern completeness
    \item \textbf{Input}: TRS $\mathcal{R}$ and defined symbol $f$
    \item \textbf{Output}: Set of constructors where $f$ is not defined    
\end{itemize}

\end{frame}

\begin{frame}{Idea}
    \begin{itemize}
        \item Start with set $M_0$ with LHSs of function $f$, set $N_0$ $f(z_1, ..., z_n)$ where $z_i$ is some constructor term 
        \item Try to unify elements of $m \in M_i$ and $n \in N_i$ with substitution $\sigma$
        \item Compute \textit{complement} of this substitution $\rho$
        \item Replace matched element in $N_i$ with new elements $n\rho$
        \item Repeat until either $M_{last}$ or $N_{last}$ is empty, or no further unification is possible
        \item If $M_{last}$ is empty but $N_{last}$ is not empty, $f$ is not defined on the terms in $N_{last}$
    \end{itemize}
\end{frame}

% \begin{frame}{Definitions}
% \begin{block}{Definition}
% The \textit{complement} of a ground term $t = c_j(t_1, ... , t_n)$ given $\mathcal{C} = \{c_1, ..., c_n\}$, $1 \leq j \leq n$ is the set $C(t)$:
% \begin{itemize}
%     \item The complement of $x \in \mathcal{X}$ is $x$
%     \item Otherwise the set $\{c_i(C(t_1), ..., C(t_n) \mid 1\leq i \leq n, i \neq j\}$
% \end{itemize}
% \end{block}

% \begin{block}{Definition}
%     The \textit{complement} of a substitution $\sigma$ is the set $C(\sigma)$ of  substitutions $\rho \neq \sigma$, with $\mathcal{D}om(\sigma) = \mathcal{D}om(\rho)$ mapping elements to complementary terms.
% \end{block}

% \end{frame}

% \begin{frame}{The algorithm}
% Initially set $M_0$ as the set of left-hand sides of $f$ and $N_0$ as the set of ground instances of $f$.
% \begin{enumerate}
%     \item Find substitution $\sigma$ to unify elements $m \in M_i$ and $n \in N_i$
%     \item Construct $M_{i+1}$ and $N_{i+1}$:
%     \begin{align*}
%     M_{i+1} &= M_{i-1} \setminus \{m\} \cup \{m\rho \mid \rho \in C(\sigma), m\rho \neq m\sigma\} \\
%     N_{i+1} &= N_{i-1} \setminus \{n\} \cup \{n\rho \mid \rho \in C(\sigma), n\rho \neq n\sigma\}
%     \end{align*}
%     \item Repeat until either $M_{last}$ or $N_{last}$ is empty, or no further unification is possible
% \end{enumerate}
% \end{frame}

% \begin{frame}{Properties}
% \begin{enumerate}
%     \item Both $M_{last}$ and $N_{last}$ are empty, the definition of $f$ is \textit{sufficiently complete}
%     \item If $M_{last}$ is empty but $N_{last}$ is not empty, $f$ is not defined on the terms in $N_{last}$
% \end{enumerate}

% Quasi-reducibility is determined when:

% \begin{itemize}
%     \item each input term is matched by the LHS of $f$ (point \textbf{1} – also pattern completeness), or
%     \item elements of $N_{last}$ can further be reduced (point \textbf{2})
% \end{itemize}
% \end{frame}

\begin{frame}{Example I - Linear case}
Constructors: $0$, $s$, function $f$: $f(\textcolor{red}{0}) \rightarrow 0$ and $f(\textcolor{red}{s}(x)) \rightarrow x$

\begin{block}{Algorithm}
\begin{center}
\begin{tabular}{ c || c | c || c | c }
    & M & N & $\sigma$ & $\rho$ \\
    \hline\hline
    0 & $f(\textcolor{red}{0}), f(s(x))$ & $f(\textcolor{red}{z})$ & $z \mapsto 0$ & $z \mapsto s(z)$ \onslide<2->\\
    1 & $f(\textcolor{red}{s}(x))$ & $f(\textcolor{red}{s}(z))$ & $z \mapsto x$ & \onslide<3->\\
    2 &  $\varnothing$ & $\varnothing$ & & \onslide<4->\\
\end{tabular}
\end{center}
\end{block}

\onslide<4->{Result: \textbf{success}, $N_2$ is empty}
\end{frame}

\begin{frame}{Example II - Incomplete pattern}
Constructors: $0$, $s$, function $f$: LHS $f(0)$ (missing case for $s(x)$).

\begin{block}{Algorithm}
\begin{center}
\begin{tabular}{ c || c | c || c | c }
    & M & N & $\sigma$ & $\rho$ \\
    \hline\hline
    0 & $f(\textcolor{red}{s}(x))$ & $f(\textcolor{red}{z})$ & $z \mapsto s(x)$ & $z \mapsto 0$ \onslide<2-> \\
    1 & $\varnothing$ & $f(0)$ & & \onslide<3->\\
\end{tabular}
\end{center}
\end{block}

\onslide<3->{Result: \textbf{failure}, $N_1$ is not empty}
\end{frame}